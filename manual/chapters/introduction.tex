\section{Introduction}

\label{Introduction} 
The current document is the user guide of the third-generation DNS code developed at the University of Magdeburg, building on top of the experience of our group, first with the PARCOMB family \cite{Parcomb,Hilbert}, then with the code called $\pi^{3}$ \cite{pi3,pi32}. Originally, the name of DINOSOARS, also called DINO, stood for "DIrect Numerical, high-Order Simulation and On-the-fly Analysis of Reacting flows and Sprays", though solid particles interacting with turbulence are now considered as well.

DNS of gaseous reactive flows has a long history of success over the last three decades, as documented for instance in \cite{oran1987, reynolds1990, poinsot1991, Baum1994, poinsot1996, vervisch1998, thevenin1995,thevenin2005}. Nevertheless, it remains a hot research topic leading steadily to new publications, e.g., \cite{hawkes2007, coussement2013,Bansal2015, Bhagatwala2015}, since many issues remain unsolved for this challenging problem involving extremely high computational requirements. Besides considering pure gaseous flames, reacting sprays have also been considered extensively by DNS, see for instance \cite{Duret2013, Komori2014}.

Our research group has always specialized in DNS studies taking into account detailed models to describe kinetic processes. As discussed in DINO paper \cite{dinosoars2015}, considering complex molecules such as $n$-heptane or ethylene,  further increases the computational burden in comparison to non-reacting flows or to approximations relying on single-step chemistry. To add to this challenge, such studies only make sense when describing with a similar level of accuracy all relevant thermodynamic and molecular transport properties, such as diffusion coefficients and viscosity. In the end, the resulting systems can only be solved on parallel supercomputers.

Finally, there is a growing need for applying DNS to configurations involving semi-complex geometries, while keeping a very low numerical dissipation. This has been answered in DINO by implementing Immersed Boundaries.

Considering the challenges that must be met and the process conditions relevant for practical purposes, three strategies have been combined in DINO to keep acceptable computational times for DNS of reacting and/or two-phase flows:
\begin{itemize}
\item Considering that most processes involve incompressible flows or low Mach numbers, only these two configurations have been considered. A large speed-up can be obtained by using a low-Mach number solver compared to a fully compressible one \cite{julian2002}.
\item The previous approach removes timestep constraints associated with acoustic waves. For reacting flows, the timestep then becomes classically controlled by fast kinetic processes. In order to solve this issue and obtain a stable, high-order integration in time, a semi-implicit time integration has been successfully implemented.
\item The third, classical strategy is to rely heavily on parallelization. For this purpose, coupling DINO with a recent open-source solution, 2DECOMP\&FFT \cite{2decomp_fft}, was found to be very efficient. However, the major issue for incompressible and low-Mach solvers is to implement a fast parallel solver for the Poisson equation coupling pressure and velocity. For this purpose, it was finally possible to develop and implement an innovative FFT-based approach, even for non-periodic boundary conditions, so that even this bottleneck could be released.

\end{itemize}

This user guide is organized as follows. After introducing how to get the code and install it, the general structure of DINO is described. Then the modules and input files are explained in detail, in case the users want to modify the code and use it for specific flow configurations. How to use the immersed boundary is introduced later. Then several quick test cases are listed and the simulations running instruction are explained. Finally, the outputs of the simulations are clearly described. Current developments of DINO are reviewed and final conclusions are obtained.
