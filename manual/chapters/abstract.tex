\newpage
\bc
\section*{Abstract}
 
\addcontentsline{toc}{section}{Abstract}

\ec

Most reacting and two-phase flows of practical interest are turbulent but take place at low Mach numbers or under incompressible conditions. In order to investigate with a high accuracy but acceptable computing times the properties of such complex flows, a suitable tool for Direct Numerical Simulations (DNS), called DINO, has been developed. The present document describes the general components and methods implemented in this code, together with the installation and running instructions of DINO.
Since applications of growing complexity are nowadays considered by DNS, a Direct Boundary Immersed Boundary Method (DB-IBM) has been implemented, allowing a description of arbitrary geometries on a fixed, but possibly refined, Cartesian mesh. A Direct Force IBM is implemented as well in DINO in order to resolve large moving spherical particles (much larger than the Kolmogorov scale) on the grid. Particles below the Kolmogorov scale are treated as point particles, taking into account additionally heat and mass transfer with the continuous flow.
The efficient parallelization of the code relies on the open-source library 2DECOMP\&FFT.
The underlying Poisson equation is solved in a fast and accurate manner by FFT, even for non-periodic boundary conditions.
The flexibility of DINO makes it a very promising tool for analyzing a variety of problems and applications involving turbulent reacting and/or two-phase flows. This user guide is not intended for people who want to really understand how the code is working (algorithms,...), but only for people who wish to use DINO to analyze turbulent reactive flow and/or two-phase flow configurations, and possibly need to make some minor changes in the code.


