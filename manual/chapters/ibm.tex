\section{Immersed boundary method}
\label{ibm}
To deal with complex geometry in the flow, DINO sticks to Cartesian grid and uses immersed boundary method (IBM). There are several different immersed boundary methods implemented in DINO, namely Direct Boundary IBM (DB-IBM), Direct Force IBM (DF-IBM) and Ghost-cell IBM.
In general, the DB-IBM is used to deal with static immersed boundaries, which will not change in time, the details of the algorithm can be found in \cite{fadlun2000, mittal2005}; the DF-IBM is used for moving particles (solid particels, droplets) resolved on the grid, the details of the algorithm can be found in \cite{lucci2010, Uhlmann2005}; the ghost-cell IBM is a higher order method which can deal with complex geometries and moving geometries.
Ghost-cell IBM is recently implemented in DINO and is still in development. The detailed algorithm can be found in \cite{Mittal2008, Chi2016}. It couples level set method for moving boundaries and it is one-way method, which means geometries can influence the flow field one way. To investigate on two phase flows, it would be better to use DF-IBM, where flows can also influence the solid particles or droplets.
Now we will introduce how to use these methods for complex geometries or moving boundries.
\subsection{Direct boundary IBM}
To use Direct boundary IBM, the users should generate their own geometry data file. (Abou, how to generate???) In the data file, every grid point has value 0 if it is within the wall, 1 if it is within the flow. There are already two examples in the RUN directory, 2D car benchmark and cylinder benchmark. Then in the input file DINO\_IN, IB\_GEOMETRY should be true, IB\_DATA should be the geometry data file. IB\_DATA\_TYPE should be 0 if the geometry data file is defined for the eulerian grid points, 1 if lagrangian data file. IB\_FLOW defines whether the flow is inside the geometry or outside the geometry.
After the input file is set properly, the users can run their cases. The DB-IBM used can only retain 0th order accuracy. Thus, it is better to use finer mesh near the geometry boundaries to obtain desirable results.
\subsection{Direct force IBM}
Direct force IBM is used to deal with two phase flow, for fully resolved, moving solid particles. Further discussion how to use???
\subsection{Ghost-cell IBM}
Ghost-cell IBM is an attempt to increase the accuracy of the IBM implemented in DINO. In the current version, ghost-cell IBM is still in development and not recommended for the users at present. If the users want to use this beta version, please run the code in one cpu core and for 2D cases. In the input file DINO\_IN, make sure IB\_GEOMETRY is turned off and IB\_GHOST is turned on. That is enough for second order ghost-cell IBM test on static complex geometries. For moving geometries, the users can pre-define the velocities of the geometries in IB\_XVEL, IB\_YVEL, IB\_ZVEL (x, y, z velocity) and IB\_WVEL (rotating velocity).
Different from DB-IBM, to use ghost-cell IBM, no external geometry data files are needed in the RUN directory. Ghost-cell IBM is coupled with level set functions to represent complex geometries accurately. The users should modify the level set function (in dino\_ib\_mod.f90) corresponding to their specific geometries. The level set function is also called distance function, which measures the distance from arbitrary point to the geometry boundaries. If the point is inside the wall, the value of the function would be negative, otherwise, positive if the point is inside the flow. Thus, the function equal to 0 represents point exactly on the boundary. Interested readers who would like to understand level set method further are refer to \cite{Osher2003, Sethian1999}. In the module file dino\_ib\_mod.f90, sphere geometry, cylinder geometry, inclined channel geometry and star geometry are already implemented in function \textit{phi} as examples.
